
%% PARA PODER HACER ETIQUETAS Y REFERENCIAS CRUZADAS USAMOS \label{key} y \ref{label} y para las ecuaciones del paquete amsmath usamos \eqref


\documentclass{article}


\usepackage[spanish]{babel}%Para indicar que autor, título, etc son títulos en castellano
\usepackage{amsmath}


\title{Dando formato al proyecto}

\author{Sergio A. Cuevas Cerón}

\date{\today}




\begin{document}
\maketitle %Inicializamos el título, autor y fecha

\begin{abstract}
	Ejemplo práctico de formato a documento en \LaTeX
\end{abstract}

\section{Introducción}
\label{sec:intro}

Más adelante en la sección \ref{sec:sec2}
Una fómula muy interesante es

\begin{equation}
\label{equ:euler}
	e^{i\pi}+1=0
\end{equation}

\subsection{Estado del arte}

Vamos a describir en que conciste la introduccón de el lenguaje latex $\ldots$

\subsection{Limitaciones de los proyectos actuales}

\subsubsection{Esta es una subsubseccion}

\section{Sección 2 del reporte}
\label{sec:sec2}

\subsection{esta es una subseccion de la seccion 2}

En este apartado abordaremos lo visto en la ecuación \ref{equ:euler}

\section{Listado de letras griegas}
\begin{itemize} %% NOS PERMITE CREAR UN LISTADO
	\item $\alpha$
	\item $\beta$
	\item $\gamma$

\end{itemize}

\end{document}