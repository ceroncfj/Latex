%Algunos sitios para encontrar recursos de LaTeX
% The latex Wikibook
%Tex Stack Exchange
%Latex Comunity
\documentclass[12pt, landscape,letterpaper]{article}
\usepackage[spanish]{babel}
\usepackage{amsmath}
\title{Curso de \LaTeX}
\author{Sergio Cuevas}
\date{\today}
\begin{document}
	
	\maketitle
	\begin{abstract}
	Aquí va el resumen Aquí va el resumen Aquí va el resumen Aquí va el resumen Aquí va el resumen Aquí va el resumen Aquí va el resumen Aquí va el resumen Aquí va el resumen Aquí va el resumen Aquí va el resumen Aquí va el resumen Aquí va el resumen Aquí va el resumen Aquí va el resumen Aquí va el resumen 
	\end{abstract}

	Para añadir las comillas: ``Es necesario comenzar con acento grave y terminar con apóstrofe"
	
	Si se quieren escribir carácteres reservados es necesario agragar un guión al inicio del símbolo
	\$
	
	In March 2006, Congress raised that ceiling an additional \$0.79 trillion to \$8.97 trillion, which is 	    approximately 68 \% of GDP. As of October 4, 2008, the``Emergency Economic Stabilization Act of 200'' raised the current debt ceiling to \$11.3 trillion.
	
	Ahora procedemos a escribir ecuaciones.
	Para escribir ecuaciones cortas basta con encerrar la expresión o símbolos entre símbolos de dólar \$
	
$a+b=c$
Para escribir ecuaciones cortas es necesario usar la función begin\{equation\}
\begin{equation}
x=\frac{-b \pm \sqrt[2]{b²-4ac}}{2a}
\end{equation}

La anterior ecuación se describe apartando un entorno especial para ello con la función begin. 

Otros entornos son itemize y enumerate

En caso de querer emplear otros paquetes será necesario declararlo al inicio del código con la función usepackage.

Con ello podemos usar funciones como las siguientes:

Para ecuaciones que no queremos que sean numeradas:

\begin{equation*}
	\Omega=\sum_{k=1}^{n} \omega_k
\end{equation*}

Hay ocasiones en las que requerimos que las letras no sean tratadas como variables multiplicadas entre sí, si no como un comando de operador matemático:
Por ejemplo

\begin{equation*}
	\min_{x,y}{x+y}
\end{equation*}


\begin{equation*}
	\frac{\operatorname{Cov}(R_i, R_m)}{\operatorname{Var}(R_m)}
\end{equation*}


Para poder alinear las operaciones con respecto a los signos igual

\begin{align*}
	(x+1)^3 &=(x+1)(x+1)(x+1)\\
			&=(x+1)(x^2+2x+1)\\
			&=(x^3+3x^2+3x+1)
\end{align*}

Ejercicio 2

Sea $X_1,X_,...,X_n$ una sucesiópn de variables aleatorias independientes e idénticamente distribuidas con $\operatorname{E}[X_i]=\mu$ i $\operatorname{Var}[X_i]=\sigma^2 < \infty$, y sea
\begin{equation*}
	S_n=\frac{1}{n}\sum_{i}^{n}X_i
\end{equation*}

su media. Entonces, cuando n tiende a infinito, las variables aliatories $\sqrt{n}(S_n-\mu)$ convergen a una distribución normal $N(0,\sigma^2)$

Atributos modificables del documento

en ``documentclass" agregamos entre corchetes los atributos

tamaño de fuente : 10pt
tamaño del papel:	a4paper
doscolumnas:	twocolumn
en forma apaisada:	landscape
a dos caras:	Twoside
características de borrador para manipulación de versión: draft


\end{document}
